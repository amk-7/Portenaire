\documentclass[12pt,a4paper]{article}
\usepackage[utf8]{inputenc}
\usepackage[french]{babel}
\usepackage[T1]{fontenc}
\usepackage{amsmath}
\usepackage{amsfonts}
\usepackage{amssymb}
\usepackage{graphicx}
\usepackage[left=2cm,right=2cm,top=2cm,bottom=2cm]{geometry}
\title{DOCUMENTATION DOCKER \textbf{Docker}}
\begin{document}
\maketitle
\section{Plan de cours}

\begin{tabular}{|c|p{12cm}|}
\hline 
Séances & chapitres \\ 
\hline 
1 & \begin{itemize}
\item[•] Généralité.
\item[•] Installer docker.
\item[•] Lancer notre première container.
\end{itemize} \\ 
\hline 
2 & \begin{itemize}
\item[•] Dockerfile : Créer une image.
\item[•] Fonctionnement du cache.
\item[•] ENTRYPOINT VS CMD.
\end{itemize} \\ 
\hline 
3 & \begin{itemize}
\item[•] Comprendre les Layers / Couches.
\item[•] Les Réseaux.
\item[•] Docker Compose.
\end{itemize} \\ 
\hline 
4 & \begin{itemize}
\item[•] Sécuriser le user namespace.
\item[•] Dokerfile multi-stage.
\item[•] Les images: TAGS, PULL ET PUSH.
\item[•] L'API Docker.
\end{itemize} \\ 
\hline 
5 & \begin{itemize}
\item[•] Développent des modules pour \textbf{PORTE-IFNTI}.
\end{itemize} \\ 
\hline 
\end{tabular} 


\end{document}



